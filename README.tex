\documentclass[a4paper,12pt]{article}
\usepackage{amsmath, amssymb, graphicx, geometry, rotating}
\geometry{margin=0.2in}

\title{MASE: Modelo de Afinidad y Sinergia Empresarial \\[1ex] \large Algoritmo de Matching entre Empresas en un Ecosistema}
\author{Rodrigo García, PhD}
\date{\today}

\begin{document}

\maketitle

\section{Descripción}
Este algoritmo identifica empresas compatibles dentro de un ecosistema empresarial basado en \textbf{objetivos estratégicos}, \textbf{número de empleados} y \textbf{ubicación geográfica}. Se generan emparejamientos según \textbf{afinidad} (similitud en objetivos) y \textbf{sinergia} (complementariedad de estrategias).

\section{Funcionalidades}
\begin{enumerate}
    \item Cargar datos desde un archivo Excel con información empresarial.
    \item Evaluar afinidad estratégica: Coincidencia en objetivos empresariales.
    \item Evaluar sinergia estratégica: Complementariedad entre empresas.
    \item Considerar el número de empleados para identificar organizaciones con tamaños similares.
    \item Considerar la ciudad para favorecer asociaciones geográficas cercanas.
    \item Generar una tabla de emparejamientos ordenada según el mejor match.
\end{enumerate}

\section{Uso}
\subsection{Carga de Datos}
El archivo debe contener las siguientes columnas:
\begin{itemize}
    \item nit: Identificación de la empresa.
    \item razonsocial: Nombre de la empresa.
    \item nombrecomercial: Nombre comercial.
    \item num\_empleados\_directos: Número de empleados directos.
    \item num\_empleados\_indirectos: Número de empleados indirectos.
    \item ciudad: Ubicación de la empresa.
    \item Crear nuevos modelos negocio, generar eficiencias, fidelizar mercado actual, diversificar mercado: Columnas binarias (1 para sí, 0 para no) que representan los objetivos estratégicos.
\end{itemize}

\section{Cálculo del Matching}
El algoritmo calcula dos tipos de emparejamientos:

\subsection{Matching por Afinidad}
Se define como el número de coincidencias exactas en los objetivos estratégicos entre dos empresas:
\begin{equation}
M_{afinidad} = \sum_{i=1}^{n} (O_{1i} = O_{2i})
\end{equation}
Donde:
\begin{itemize}
    \item $O_{1i}$ y $O_{2i}$ son los valores de los objetivos estratégicos de las empresas 1 y 2.
    \item $n$ es el número total de objetivos estratégicos evaluados.
    \item Se suma 1 por cada objetivo coincidente.
\end{itemize}

\subsection{Matching por Sinergia}
Evalúa qué tan complementarias son dos empresas, es decir, cuando una tiene un objetivo que la otra no tiene:
\begin{equation}
M_{sinergia} = \sum_{i=1}^{n} ((O_{1i} = 1) \land (O_{2i} = 0)) + \sum_{i=1}^{n} ((O_{1i} = 0) \land (O_{2i} = 1))
\end{equation}

\subsection{Diferencia de Empleados}
Se usa una normalización inversa para favorecer empresas con números similares de empleados:
\begin{equation}
M_{empleados} = \frac{1}{1 + |E_1 - E_2|}
\end{equation}
Donde:
\begin{itemize}
    \item $E_1$ y $E_2$ son el número total de empleados de cada empresa.
    \item Se suma 1 en el denominador para evitar divisiones por cero.
    \item Empresas con tamaños similares obtienen valores más altos.
\end{itemize}

\subsection{Coincidencia de Ciudad}
Si ambas empresas están en la misma ciudad:
\begin{equation}
M_{ciudad} = \begin{cases} 
1, & \text{si la ciudad es la misma} \\
0, & \text{si son ciudades diferentes} 
\end{cases}
\end{equation}

\subsection{Puntaje Total de Matching}
El puntaje total se obtiene combinando todas las métricas anteriores:
\begin{equation}
M_{total} = M_{afinidad} + M_{sinergia} + M_{empleados} + M_{ciudad}
\end{equation}

\section{Interpretación de los Resultados}
La salida consiste en una tabla con:
\begin{itemize}
    \item \texttt{Empresa 1} y \texttt{Empresa 2}: Empresas emparejadas.
    \item \texttt{Ciudad 1} y \texttt{Ciudad 2}: Ubicación de cada empresa.
    \item \texttt{Match Afinidad}: Puntuación de coincidencia estratégica.
    \item \texttt{Match Sinergia}: Puntuación de complementariedad estratégica.
    \item \texttt{Total Empleados 1} y \texttt{Total Empleados 2}: Cantidad total de empleados por empresa.
    \item \texttt{Diferencia Empleados}: Diferencia en número de empleados.
\end{itemize}

\section{Ejemplo de Resultados}
\begin{table}[h]
\centering
\begin{tabular}{|p{2.5cm}|p{2.5cm}|p{2cm}|p{2cm}|p{1.5cm}|p{1.5cm}|p{1.5cm}|p{1.5cm}|}
    \hline
    Empresa 1 & Empresa 2 & Ciudad 1 & Ciudad 2 & Match Afinidad & Match Sinergia & Total Empleados 1 & Total Empleados 2 \\
    \hline
    E1 & E2 & Valledupar & Valledupar & 6 & 2 & 95 & 95 \\
    E3 & E4 & Riohacha & Riohacha & 6 & 2 & 80 & 80 \\
    E5 & E6 & Sincelejo & Sincelejo & 6 & 2 & 140 & 140 \\
    \hline
\end{tabular}
\caption{Tabla de resultados en orientación horizontal.}
\end{table}

\section{Aplicaciones y Uso}
\begin{itemize}
    \item \textbf{Fomentar asociaciones estratégicas}: Empresas con alta afinidad pueden formar alianzas para potenciar sus estrategias.
    \item \textbf{Identificar oportunidades de crecimiento}: Empresas con alta sinergia pueden complementarse en el mercado.
    \item \textbf{Optimizar ecosistemas empresariales}: Facilita la integración de empresas en zonas geográficas estratégicas.
\end{itemize}

\end{document}